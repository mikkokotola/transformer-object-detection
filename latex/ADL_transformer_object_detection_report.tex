%% This file is modified by Jussi Kangasharju, Pirjo Moen, and Veli
%% Mäkinen from HY_fysiikka_LuKtemplate.tex authored by Roope Halonen ja
%% Tomi Vainio. Some text is also inherited from engl_malli.tex by
%% Kutvonen, Erkiö, Mäkelä, Verkamo, Kurhila, and Nykänen.
%% 
%% 

% STEP 1: Choose oneside or twoside
\documentclass[english,oneside,openright]{HYreportMLDS}
%finnish,swedish
%
% Own packages
\usepackage{verbatim}

%\usepackage[utf8]{inputenc} 
% For UTF8 support. Use UTF8 when saving your file.
\usepackage{lmodern} % Font package 
\usepackage{textcomp} % Package for special symbols 
\usepackage[pdftex]{color, graphicx} % For pdf output and jpg/png graphics 
\usepackage[pdftex, plainpages=false]{hyperref} % For hyperlinks and pdf metadata 
\usepackage{fancyhdr} % For nicer page headers 
\usepackage{tikz} % For making vector graphics (hard to learn but powerful)
%\usepackage{wrapfig} % For nice text-wrapping figures (use at own discretion)
\usepackage{amsmath, amssymb} % For better math
%\usepackage[square]{natbib} % For bibliography
\usepackage[footnotesize,bf]{caption} % For more control over figure captions 
\usepackage{blindtext} 
\usepackage{titlesec}
\usepackage[titletoc]{appendix}

\onehalfspacing %line spacing
%\singlespacing 
%\doublespacing

%\fussy 
\sloppy % sloppy and fussy commands can be used to avoid overlong text lines

% STEP 2: 
% Set up all the information for the title page and the abstract form. 
% Replace parameters with your information.
\title{End-to-end Object Detection with Transformers, DRAFT}
\author{Mikko Kotola and Juuso Lassila}
\date{\today}
%\prof{Professor X or Dr. Y} 
%\censors{Professor A}{Dr. B}{}
\keywords{machine learning, x} \depositeplace{}
\additionalinformation{}


\classification{\protect{\ \\
    \ Computing methodologies $\rightarrow$ Machine learning $\rightarrow$ Machine learning algorithms \\
}}

% if you want to quote someone special. You can comment this line and
% there will be nothing on the document.
%\quoting{Bachelor's degrees make pretty good placemats if you get them
%laminated.}{Jeph Jacques}

% OPTIONAL STEP: Set up properties and metadata for the pdf file that
% pdfLaTeX makes. But you don't really need to do this unless you want
% to.
\hypersetup{ 
	bookmarks=true,         % show bookmarks bar first?
	unicode=true,           % to show non-Latin characters in Acrobat’s bookmarks 
	pdftoolbar=true,        % show Acrobat’s toolbar?
	pdfmenubar=true,        % show Acrobat’s menu? 
	pdffitwindow=false,		% window fit to page when opened 
	pdfstartview={FitH},    % fits the width of the page to the window 
	pdftitle={},            % title
	pdfauthor={},           % author 
	pdfsubject={},          % subject of the document 
	pdfcreator={},          % creator of the document
	pdfproducer={pdfLaTeX}, % producer of the document
	pdfkeywords={}, % list of keywords for
	pdfnewwindow=true,      % links in new window 
	colorlinks=true, 		% false: boxed links; true: colored links 
	linkcolor=black,        % color of internal links 
	citecolor=black,        % color of links to bibliography 
	filecolor=magenta,      % color of file links 
	urlcolor=cyan			% color of external links
}
          
\begin{document}

% Generate title page.
\maketitle

% STEP 3: Write your abstract (of course you really do this last). You
% can make several abstract pages (if you want it in different
% languages), but you should also then redefine some of the above
% parameters in the proper language as well, in between the abstract
% definitions.

%\begin{abstract} 
% Abstract stuff.
%\end{abstract}

% Place ToC
\mytableofcontents

\mynomenclature

% ----------------------------------------------------------------------
% STEP 4: Write the thesis. Your actual text starts here.
% You shouldn't mess with the code above the line except to change the
% parameters. Removing the abstract and ToC commands will mess up stuff.

\chapter{Introduction}
\label{chapter:intro}

Intro.
End-to-end object detection~\cite{carion2020endtoend}.

COCO object detection dataset~\cite{Coco} augmented with stuff categories~\cite{CocoStuff}.

ADE20K dataset~\cite{zhou2017scene}. 

\chapter{Xxx}
\label{chapter:Xxx}


\chapter{Conclusions}
\label{chapter:conclusions}
Wrap it up. 


% STEP 5: Uncomment the following lines and set your .bib file and
% desired bibliography style to make a bibliography with BibTeX.
% Alternatively you can use the thebibliography environment if you want
% to add all references by hand.
% 
\cleardoublepage %fixes the position of bibliography in bookmarks
\phantomsection

\renewcommand\bibname{References}
\addcontentsline{toc}{chapter}{\bibname} % This lines adds the bibliography to the ToC 
\bibliographystyle{abbrv} % numbering alphabetic order 
\bibliography{references}

\begin{comment}
Comment sections can be used to leave parts out.
\end{comment}

\end{document}